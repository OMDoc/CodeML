\documentclass[11pt,twoside,titlepage]{report}
\usepackage{array,longtable} %standard
\usepackage{listings,lstomdoc,acronyms}  
\usepackage[today,fancyhdr]{svninfo}
\usepackage{macros}   %local
\usepackage[a4paper=true,  bookmarks=true, linkcolor=black,
citecolor=black,urlcolor=black,colorlinks=true,pagecolor=black,
breaklinks=true, bookmarksopen=true]{hyperref}
\makeindex
\edshownotes
\title{{\codeml}: An Open Markup Format the Content and Presentatation of Program Code}
\author{Michael Kohlhase\\
  Computer Science, Carnegie Mellon University\\
  Pittsburgh, Pa 15213, USA\\
  {\url{http://www.cs.cmu.edu/~kohlhase}}}

\begin{document}
\svnInfo $Id: codeml.tex 15 2008-02-22 07:55:51Z kohlhase $
\svnKeyword $HeadURL: https://svn.omdoc.org/repos/codeml/doc/spec/codeml.tex $

\maketitle
\setcounter{page}{0}\pagenumbering{roman}
\begin{abstract}
  This specification defines the Code Markup Language, or {\codeml}. {\codeml} is
  an {\xml} application for describing progra codem and capturing both its
  structure and content. The goal of {\codeml} is to enable program code to be
  served, received, and processed on the World Wide Web, just as HTML has enabled
  this functionality for text.
  
  {\codeml} is inspired by the {\mathml} and {\openmath} format for markup of
  mathematical formulae and can be used in conjuction with these formats in a
  variety of document formats.

  \paragraph{Status of this Document:} 

  This specification of the markup language {\codeml} is intended primarily for a
  readership consisting of those who will be developing or implementing renderers
  or editors using it, or software that will communicate using {\codeml} as a
  protocol for input or output. It is not a User's Guide but rather a reference
  document.
  
  The current report is an early draft of the specification of the {\codeml}
  format. Even though the basic layout of the format should survive, details like
  element- or attribute names, and content models are very likely to change as
  implementation and application experience grows.
\end{abstract}
\newpage
\setcounter{tocdepth}{2}\tableofcontents
newpage\pagenumbering{arabic}
\svnInfo $Id: intro.tex 15 2008-02-22 07:55:51Z kohlhase $
\svnKeyword $HeadURL: https://svn.omdoc.org/repos/codeml/doc/spec/intro.tex $
\chapter{Introduction}\label{sec:intro} 

This specification defines the {\indextoo{Code Markup
    Language}}\index{markup!language! for program code}, or {\codeml}.  {\codeml} is an
{\xml} application for describing program code and capturing both its structure and
content. The goal of {\codeml} is to enable program code to be served, received,
and processed on the World Wide Web, just as HTML has enabled this functionality
for text.
  
The {\em primary purpose} of {\indextoo{program}} code\index{code!program} is as a
means of communication between the {\indextoo{programmer}} and the machine as a
vehicle to express programs that can be executed by the machine. For this purpose,
the field of {\indextoo{computer science}} has developed a multitude of
{\indextoo{programming language}s}, i.e. {\indextoo{formal
    language}s}\index{language!formal} with a well-defined structure and fixed
semantics that allow the programmer to express her operational intentions as
{\indextoo{texts}} called {\indextoo{program code}}. This code is then interpreted
by a suitable programs (e.g. a {\indextoo{compiler}}) and translated into native
instructions to the computer, which are then executed to achieve the desired
result.

An {\em auxiliary purpose} of program code is to communicate about
{\indextoo{algorithm}s} and programs between humans, e.g. in {\indextoo{program
    development}}\index{development!of programs} {\indextoo{documentation}},
{\indextoo{quality assurance}}, or computer science {\indextoo{education}}. To
allow humans to cope with the complexities of program code, the program
development process is supported by a variety of ``intelligent development
environments''\index{development!environment}, which provide features like code
{\indextoo{pretty-printing}} (e.g. syntax {\indextoo{highlighting}},
{\indextoo{indentation}}), {\indextoo{hyper-linking}}, and even higher-level
techniques like {\indextoo{literate programming}}, or {\indextoo{documentation}}
{\indextoo{extraction}} from {\indextoo{commented code}}\index{code!commented}.

Unfortunately, this functionality is directly tied to the development environment
and (usually) to a particular programming language: The program text has to be
analyzed, and the structural and syntactic information has to be conveyed to the
user in a dedicated user interface. It is currenly only possible to export the
functionality of these development environments to the World Wide Web to a very
limited extent. 
\begin{itemize}
\item Web markup formats like {\html} only provide limited primitives for
  pretty-printing program listings; the possibilities are largely limited to using
  the
  {\tt{pre}}\index{element!{\tt{pre}}(html)}\index{html!{\tt{pre}}}\index{pre@{\tt{pre}}
    (html)} element, {\indextoo{table}s}, or non-breaking spaces {\ttin{\&nbsp;}}.
  Together with CSS styling this is used for (static, fixed-width) indenting and
  syntax high-lighting.
\item Using technologies like {\xsl} formatting objects (which allow
  flexible-width formatting\ednote{check that they do}) is problematic, since the
  syntactic and structural information necessary for pretty-printing is implicit
  in the program structure and a full parser that is needed to reveal it,
  transcends the possibilities of both {\ttin{javascript}} and and {\xslt} (which
  are the computational engines built into current browsers). Moreover, an
  approach based on parsing programming languages would necessitate the
  availability of gramars for all programming languages. 
\item As programming languages are optimized for compilation and execution, they
  do not supply an infrastructure for referencing program fragments, which would
  be\ednote{continue}
\end{itemize}

\ednote{say something more, dynamics, local services without
  program interpretation. This is similar to the situation to math formulae, say
  something about what we can do with {\mathml}}

The concrete design of the {\codeml} format is inspired by the
{\mathml}\index{mathml@{\mathml}} (the {\indextoo{Mathematical Markup Language}})
and {\openmath} format for markup of mathematical formulae\index{markup!language!
  for mathematical formulae}\index{formulae!mathematical}\index{mathematical
  formulae} and can be used in conjuction with these formats in a variety of
document formats.


The most relevant related work is probably the {\tt{listings.sty}}
package~\cite{Heinz:tlp02} for {\LaTeX}, which allows to mark up program listings
for print presentation.\ednote{say some more, maybe move somewhere else}

\ednote{given an overview over what there already is.}

%%% Local Variables: 
%%% mode: latex
%%% TeX-master: "codeml"
%%% End: 

\svnInfo $Id: format.tex 15 2008-02-22 07:55:51Z kohlhase $
\svnKeyword $HeadURL: https://svn.omdoc.org/repos/codeml/doc/spec/format.tex $
\chapter{{\codeml} Elements}\label{chap:codeml}

The {\codeml} format is loosely modeled after the {\mathml}
format~\cite{CarIon:MathML01} format. It has three kinds of elements, which we
will specify in the rest of the section: {\indextoo{presentation}}
elements\index{element!presentation} (section~\ref{sec:presentation}), {\indextoo{content}}
elements\index{element!content} (section~\ref{sec:content}), and {\indextoo{integration}}
elements\index{element!integration} (section~\ref{sec:integration}).

Since the {\codeml} fragment is intended as a module for the
{\omdoc}~\cite{Kohlhase:omfmd01} format, it only needs to be concerned with
representing the actual code and can re-use the library- and theory levels of
{\omdoc}\ednote{I hope}.

The {\codeml} presentation elements are used to mark up the syntactic structure of
program source code. They classify and group sub-strings of code in arbitrary
programming languages by their syntactic function. As a consequence, generic tools
can perform added-value services, without requiring parsers for the
code.\ednote{continue, say something about content, interaction.}

\section{Presentation Elements}\label{sec:presentation}

In this section we will introduce the {\codeml} presentation elements, which can
be used to classify and group sub-strings of program code by their syntactic
function. 

All presentation {\codeml} elements have the following atrributes in common: 
\begin{description}
\item[\attribute{xlink:xref}{presentation-CodeML}] This optional attribute
  specifies a cross-reference to a content-{\codeml} element that corresponds to
  the code contained in this element. This attribute (together with the next two)
  conforms to the W3C {\xlink} recommendation~\cite{DeRMal:xlink01} and is
  recognized by many {\xml} applications. 
\item[\attribute{xlink:type}{presentation-CodeML}] This attribute must have the
  value {\attval{simple}{xlink:type}{presentation-CodeML}}, according to the
  {\xlink} recommendation. It specifies that
  {\attribute{xlink:xref}{presentation-CodeML}} is a so-called {\indextoo{simple
      link}}\index{link!simple}, like the {\element{a}} element from {\html}. The
  DTD (see appendix~\ref{sec:dtd}) fixes this attribute, so in {\codeml} documents
  with DTD, it need not be explicitly specified.
\item[\attribute{xlink:role}{presentation-CodeML}] This attribute specifies the
  role of link established by the {\attribute{xlink:xref}{presentation-CodeML}} as
  a semantic equivalence. Like the {\attribute{xlink:type}{presentation-CodeML}},
  the DTD fixes its value.
\end{description}

We will use the code fragment for a {\java} implementation of the {\ttin{find}}
procedure in {\mylstref{find-raw}} as a running example in this section.  The
presentation-{\codeml} equivalent of this presented in
{\mylstref{find-presentation}}.

\begin{lstlisting}[float,frame=lines,label=lst:find-raw,
   language=java,numbers=none,
   caption={A java code snippet},
   index={find}]
public int find (int x) {
    if (s[x] < 0) return x;
    return  find (s[x]);
    } 
\end{lstlisting}

\begin{lstlisting}[float,frame=lines,label=lst:find-presentation,
   caption={The {\codeml} presentation for the code snippet in {\mylstref{find-raw}}},
   index={[2]cpg,cpo,cptype,code}]
<cpg> 
  <cpg>
    <cpo>public</cpo><cptype>int</cptype><cpo>find</cpo>
    <cpg open="(" close=")"><cptype>int</cptype><cpi>x</cpi></cpg>
  </cpg>
  <cpg open="{" close="}" breakO="hard">
    <cpo>if</cpo>
    <cpg open="(" close=")">
      <cpo>s</cpo>
      <cpg open="[" close="]"><cpi>x</cpi></cpg>
      <cpo>&lt;</cpo>
      <cpb type="number">0</cpb>
    </cpg>
    <cpg close=";" Cbreak="hard"><cpo>return</cpo><cpi>x</cpi></cpg>
    <cpg close=";" Cbreak="hard">
      <cpo>return</cpo>
      <cpg>
        <cpo>find</cpo>
        <cpg open="(" close=")">
          <cpo>s</cpo>
          <cpg open="[" close="]"><cpi>x</cpi></cpg>
        </cpg>
      </cpg>
    </cpg>
  </cpg>
</cpg>
\end{lstlisting}

\subsection{Token Elements}\label{sec:pres:token}

Token elements in presentation markup are broadly intended to represent the
smallest code fragments which carry meaning. Thus character data in {\codeml}
markup is only allowed to occur as part of the content of token elements. The only
exception is whitespace between elements, which is ignored. Token elements can
contain any sequence of zero or more Unicode characters.

Token elements should be rendered as their content (i.e. in the visual case, as a
closely-spaced horizontal row of standard glyphs for the characters in their
content). Rendering algorithms should also take into account the style attributes
as described below, and modify surrounding spacing by rules or attributes specific
to each type of token element.

{\codeml} has the following three elements to classify token elements; all of
these can contain arbitrary unicode text\footnote{and the {\xml} elements
  specified by changing the {\ttin{\&cpt.extra.content;}} parameter entity either
  in the internal subset of the DOCTYPE declaration or in the DTD driver}
\begin{description}
\item[\eldef{cpb}] for {\indextoo{basic object}s}\index{object!basic}. This
  element should be used for values like {\indextoo{number}s},\ednote{Maybe better
    call it {\tt{cpv}}?}. It has the optional attribute {\attribute{type}{cpb}},
  which can take the values {\attval{number}{type}{cpb}}\ednote{what else?} to
  classify its contents.
\item[\eldef{cpo}] for {\indextoo{operator}s}, i.e. names of symbols with a fixed
  meaning. These include the {\indextoo{built-in}}\index{operator!built-in}
  operators of the programming language, functions that are
  {\indextoo{imported}}\index{operator!imported} from a {\indextoo{library}}, and
  {\indextoo{defined}}\index{operator!defined} ones.  The optional
  {\attribute{type}{cpo}} supplies the values {\attval{built-in}{type}{cpo}},
  {\attval{imported}{type}{cpo}}, and {\attval{defined}{type}{cpo}} for these
  cases.\ednote{are there more? yes: recursive-call and definiiens}
\item[\eldef{cpi}] for {\indextoo{identifiers}} in the programming language,
  these are usually variable names\ednote{what else?}.
\item[\eldef{cptype}] for {\indextoo{data type}s}\index{type!data}\ednote{do we
    want this as a separate thing, or can we subsume them into {\element{cpo}}}.
\end{description}
The classification supported by these elements can be used by rendering engines
e.g. to highlight\index{highlighting}\index{code!highlighting} or
{\indextoo{color}}\index{code!coloring} program code.

\begin{myfig}{token}{The {\codeml} Token Elements}
  \quicktable{\tokentable{}}
\end{myfig}

\subsection{Token Element Attributes}\label{sec:token-attributes}
Until full {\indextoo{Unicode}} support is widely available, {\codeml} provides
four code style attributes (see {\myfigref{style-attrib}}). These attributes are
valid on all presentation token elements and on no other elements except
{\element{cstyle}}.

\begin{myfig}{style-attrib}{{\codeml} Style Attributes}
\begin{tabular}{|p{2cm}|p{7.5cm}|l|}\hline
Name &  values & default\\\hline\hline
{\attribute{variant}{p-{\codeml}}} 
   & {\attval{normal}{variant}{p-{\codeml}}}, 
     {\attval{bold}{variant}{p-{\codeml}}}, 
     {\attval{italic}{variant}{p-{\codeml}}}, 
     {\attval{bold-italic}{variant}{p-{\codeml}}}, 
     {\attval{double-struck}{variant}{p-{\codeml}}}, 
     {\attval{bold-fraktur}{variant}{p-{\codeml}} }, 
     {\attval{script}{variant}{p-{\codeml}}}, 
     {\attval{bold-script}{variant}{p-{\codeml}}}, 
     {\attval{fraktur}{variant}{p-{\codeml}}}, 
     {\attval{sans-serif}{variant}{p-{\codeml}}}, 
     {\attval{bold-sans-serif}{variant}{p-{\codeml}}},
     {\attval{sans-serif-italic}{variant}{p-{\codeml}}}, 
     {\attval{sans-serif-bold-italic}{variant}{p-{\codeml}}}, 
     {\attval{monospace}{variant}{p-{\codeml}}} 
   & normal\\\hline
{\attribute{size}{p-{\codeml}}} 
   & {\attval{small}{size}{p-{\codeml}}}, 
     {\attval{normal}{size}{p-{\codeml}}}, 
     {\attval{big}{size}{p-{\codeml}}}, 
     {\attval{number}{variant}{p-{\codeml}}},
     {\attval{v-unit}{variant}{p-{\codeml}}}
   & inherited\\\hline
{\attribute{color}{p-{\codeml}}}, {\attribute{background}{p-{\codeml}}}    
   & {\attval{{\#}rgb}{color,background}{p-{\codeml}}}, 
     {\attval{{\#}rrggbb}{color,background}{p-{\codeml}}}, 
     {\attval{aqua}{color,background}{p-{\codeml}}},
     {\attval{black}{color,background}{p-{\codeml}}},
     {\attval{blue}{color,background}{p-{\codeml}}},
     {\attval{fuchsia}{color,background}{p-{\codeml}}},
     {\attval{gray}{color,background}{p-{\codeml}}},
     {\attval{green}{color,background}{p-{\codeml}}},
     {\attval{lime}{color,background}{p-{\codeml}}},
     {\attval{maroon}{color,background}{p-{\codeml}}},
     {\attval{navy}{color,background}{p-{\codeml}}},
     {\attval{olive}{color,background}{p-{\codeml}}},
     {\attval{purple}{color,background}{p-{\codeml}}},
     {\attval{red}{color,background}{p-{\codeml}}},
     {\attval{silver}{color,background}{p-{\codeml}}},
     {\attval{teal}{color,background}{p-{\codeml}}},
     {\attval{white}{color,background}{p-{\codeml}}},
     {\attval{yellow}{color,background}{p-{\codeml}}}
     & inherited\\\hline
  \end{tabular}
\end{myfig}
 
The {\indextoo{style attribute}s}\index{attribute!style} define logical classes of
token elements. Each class is intended to correspond to a collection of
typographically-related symbolic tokens that have a meaning within a given math
expression, and therefore need to be visually distinguished and protected from
inadvertent document-wide style changes which might change their meanings.

When {\codeml} rendering takes place in an environment where CSS is available, the
style attributes can be viewed as predefined selectors for CSS style rules. When
CSS is not available, it is up to the internal style mechanism of the rendering
application to visually distinguish the different logical classes.

Token elements also permit {\attribute{id}{*}}, {\attribute{class}{*}} and
{\attribute{style}{*}} attributes for compatibility with style sheet
mechanisms.\ednote{rethink; put this into DTD}

Since {\codeml} expressions are often embedded in a textual data format such as
XHTML, the surrounding text and the {\codeml} must share rendering attributes such
as font size, so that the renderings will be compatible in style. For this reason,
most attribute values affecting text rendering are inherited from the rendering
environment, as shown in the `default' column in the table above. 

The {\attribute{color}{p-{\codeml}}} attribute controls the color in which the
content of tokens is rendered. Additionally, when inherited from
{\element{cstyle}}, it controls the color of all other drawing by {\codeml}
elements.

The values of {\attribute{color}{p-{\codeml}}}, and
{\attribute{background}{p-{\codeml}}} can be specified as as an {\html} color
name\footnote{Note that the color name keywords are not case-sensitive, unlike
  most keywords in {\codeml} attribute values for compatibility with CSS and
  {\html}.}  or as a string consisting of '\#' followed without intervening
whitespace by either 1-digit or 2-digit hexadecimal values for the red, green, and
blue components, respectively, of the desired color, with the same number of
digits used for each component (or as the keyword `transparent' for background).
The hexadecimal digits are not case-sensitive. The possible 1-digit values range
from 0 (component not present) to F (component fully present), and the possible
2-digit values range from 00 (component not present) to FF (component fully
present), with the 1-digit value x being equivalent to the 2-digit value xx
(rather than x0).

The color syntax described above is a subset of the syntax of the color and
background-color properties of CSS. The background-color syntax is in turn a
subset of the full CSS background property syntax, which also permits
specification of (for example) background images with optional repeats. The more
general attribute name background is used in {\codeml} to facilitate possible
extensions to the attribute's scope in future versions of {\codeml}.

\subsection{Grouping, Indenting and Breaking}\label{sec:pres:grouping}

One of the most important {\codeml} uses of the information in presentation
{\codeml} markup is to indent\index{indentation} and linebreak\index{linebreaking}
program code. Conventionally, this is hardcoded by {\indextoo{tabstop}s} and
{\indextoo{space}s} in the source code, which is sufficient for program
development, but carries implicit assumptions about screen size and output format,
which are not sustainable in the Internet and ubitiquos computing age (just
imagine debugging lisp code on a palmtop). Therefore {\codeml} relies on the
concept of ``semantic'' concept of code grouping to convey the syntactical
structure of programming language expressions, and relies on style sheets or
presentation engine to translate this into visual aids, such as indentation, or
folding. 

{\codeml} uses the {\eldef{cpg}} element to group program code fragments, as a
consequence, it can contain any (number of) presentation {\codeml} elemements.
Since grouping is often supported by bracketing structures in programming langues,
the {\element{cpg}} element has three optional attributes {\attribute{open}{cpg}}
and {\attribute{close}{cpg}} which specify the {\indextoo{opening}} and
{\indextoo{closing}} {\indextoo{brackets}} used the presented code. The children
of the group are separated by single spaces, unless this is overridden by setting
the optional {\attribute{separator}{cpg}} attribute to something else (e.g. a
linefeed).  Furthermore, the {\element{cpg}} element has an optional
{\attribute{indent}{cpg}} attribute, which allows the author to specify a factor
for the default indentation increment used by the presentation agent. Finally, the
{\element{cpg}} element has six attributes that control the linebreaking. They
come in three groups of pairs one each for opening, closing brackets and
separators. {\attribute{breakO}{cpg}} specifies the break before the opening
bracket, and {\attribute{Obreak}{cpg}} a break after it. Attributes
{\attribute{breakC}{cpg}} and {\attribute{Cbreak}{cpg}} to the analogous for the
closing brackets, while {\attribute{breakS}{cpg}} and {\attribute{Sbreak}{cpg}}
address the breaks for the separators\footnote{Note that we cannot specify a
  breaking strategy for individual separator positions; all are treated alike.}.
Currently, {\codeml} does not restrict the values of these attributes leaving the
specifics to applications; we suggest using the keyword {\ttin{hard}} for ``hard''
(unconditional) {\indextoo{break}s}. A system of numbers might be used for
{\index{prioritized break}s}.\ednote{give/discuss examples}

Even if the ultimate linebreaking of the code presentation has to be determined by
the presentation agent, {\codeml} provides the empty {\eldef{cpbr}} element that
allows the author to force a line break even if the presentation agent does not
see the need for one on the basis of the grouping information.\ednote{do we really
  need this}
\begin{myfig}{group}{The {\codeml} Grouping Elements}
  \quicktable{\grouptable{}}
\end{myfig}

\subsection{Descriptive Text and Comments}\label{sec:pres:text}

Many programs contain descriptive text, either as documentation, comments, or (in
the case of pseudo-code) even as part of the ``programming language''. {\codeml}
provides the elements {\eldef{cpd}} for {\indextoo{descriptive
    text}}\index{text!descriptive} and {\eldef{cpc}} for {\indextoo{comment}s}.
Both elements can contain a {\indextoo{multilingual
    group}}\index{group!multilingual} of {\eldef{cpt}} elements that contain the
actual natural language text. 

\begin{myfig}{text}{The {\codeml} Text Elements}
  \quicktable{\texttable{}}
\end{myfig}

{\element{cpt}} elements have an {\ttin{xml:lang}} attribute that specifies the
language they are written in. Thus using multilingual groups\index{multilingual
  support}\index{support!multilingual}\index{languages!multiple} of
{\element{cmp}} elements with different languages can promote {\codeml}
internationalization.  Conforming with the {\xml} recommendation, we use the
{\indextoo{ISO 639}} two-letter {\indextoo{country code}s}\index{code!country}
({\ttin{en}}$\;\widehat=\;$English, {\ttin{de}}$\;\widehat=\;$German,
{\ttin{fr}}$\;\widehat=\;$French, {\ttin{nl}}$\;\widehat=\;$Dutch,\ldots).  This
optional attribute has the default ``{\attval{en}{xml:lang}{*}}'', so that if no
{\attribute{xml:lang}{*}} is given, then English is assumed. Of course it is
forbidden to have more than one {\element{cpt}} per value of
{\attribute{xml:lang}{cpt}} per {\element{cpd}} or {\element{cpc}} element,
moreover, {\element{cpt}}s that are siblings must be translations of each other.

\begin{lstlisting}[float,frame=lines,label=lst:find-pseudo,escapechar=\%,
   caption={Pseudo-code for the code snippet in {\mylstref{find-raw}}},
   index={[2]cpc,cpd,cpt}]
 <code format="pseudo-code">
 <cpg>
   <cpo>procedure</cpo>
   <cpg indent="0">
     <cpi>find</cpi>
     <cpg open="(" close=")"><cpi>x</cpi></cpg>
     <cpg ="break" breaks="bo-c">
       <cpg>
         <cpo>if</cpo>
         <cpd>%\label{lst-qr:lst:find-pseudo:cpd}%
           <cpt>s[x] is negative</cpt>
           <cpt xml:lang="de">s[x] ist negativ</cpt>
         </cpd>
       </cpg>
       <cpg>
         <cpo>then</cpo>
         <cpo>return</cpo>
         <cpi>x</cpi>
         <cpc>%\label{lst-qr:lst:find-pseudo:cpc}%
           <cpt>we are at the root</cpt>
           <cpt xml:lang="de">Wir sind in der Wurzel</cpt>
         </cpc>
       </cpg>
       <cpg>
         <cpo>otherwise</cpo>
         <cpo>return</cpo>
         <cpi>s[x]</cpi>
         <cpc>
           <cpt>way to go!</cpt>
           <cpt xml:lang="de">auf gehts!</cpt>
         </cpc>
       </cpg>
     </cpg>
   </cpg>
 </cpg>
</code>
\end{lstlisting}

Let us consider an example: {\mylstref{find-pseudo}} shows some
presentation-{\codeml} for our running example from {\mylstref{find-raw}}. This
example is mildly internationalized: in line {\ref{lst-qr:lst:find-pseudo:cpd}},
we have a {\element{cpd}} element and in line {\ref{lst-qr:lst:find-pseudo:cpc}},
we have a comment; both English and German text. A suitable presentation engine
could generate {\indextoo{localized}} presentations like the ones in 
{\myfigref{find-pseudo-en-de}} from the source in
Listing{\ref{lst:find-pseudo}}.\ednote{can the text contain code snippets again? Update
  the DTD and the examples to allow for that.}

\begin{myfig}{find-pseudo-en-de}{Multi-lingual Pseudo-code presentation 
    from {\mylstref{find-pseudo}}} \def\bu#1{{\bf\underline{#1}}}
  \fbox{\begin{minipage}{9cm}
      {\bu{procedure}} find (x) \\
      \hspace*{2em}    {\bu{if}} {\sf s[x] is negative}\\
      \hspace*{4em}     {\bu{then}} {\bu{return}} x\hfill ({\sl we are at the root})\\
      \hspace*{4em} {\bu{otherwise}} {\bu{return}} s[x]\hfill ({\em way to go!})
\end{minipage}}
\fbox{\begin{minipage}{9cm}
{\bu{procedure}} find (x) \\
\hspace*{2em}    {\bu{if}} {\sf s[x] ist negativ}\\
\hspace*{4em}     {\bu{then}} {\bu{return}} x\hfill ({\sl Wir sind an der Wurzel})\\
\hspace*{4em}     {\bu{otherwise return}} s[x]\hfill ({\sl auf gehts!})
\end{minipage}}
\end{myfig}

\subsection{Raw Code}\label{sec:raw}

To facilitate migration from raw code and to allow for partial markup, {\codeml}
allows to include\issue{do we also want to allow this in content?} {\indextoo{raw
    code}}\index{code!raw} in the {\element{cpr}} element\issue{collapse this with
  the {\element{rawcode}} element?}.\ednote{table?}

\subsection{Style Change}\label{sec:style}

The {\element{cstyle}} element is used to make style changes that affect the
rendering of its contents. {\element{cstyle}} can be given any attribute accepted
by any {\codeml} presentation element provided that the attribute value is inherited,
computed or has a default value; presentation element attributes whose values are
required are not accepted by the mstyle element. In addition mstyle can also be
given certain special attributes listed below.

The {\element{cstyle}} element accepts any number of arguments. If this number is
not 1, its contents are treated as a single `inferred {\element{cpg}}' formed from
all its arguments.

Loosely speaking, the effect of the {\element{cstyle}} element is to change the
default value of an attribute for the elements it contains. Style changes work in
one of several ways, depending on the way in which default values are specified
for an attribute.  The cases are:\ 
\begin{itemize}
\item Some attributes, such as {\attribute{size}{p-{\codeml}}} or
  {\attribute{color}{p-{\codeml}}}, are inherited from the surrounding context when
    they are not explicitly set.  Specifying such an attribute on an
    {\element{cstyle}} element sets the value that will be inherited by its child
    elements. Unless a child element overrides this inherited value, it will pass
    it on to its children, and they will pass it to their children, and so on. But
    if a child element does override it, the new (overriding) value will be passed
    on to that element's children, and then to their children, etc, until it is
    again overridden.
  \item Other attributes, such as {\attribute{variant}{p-{\codeml}}}, have default
    values that are not normally inherited. That is, if the
    {\attribute{variant}{p-{\codeml}}} attribute is not set on an element, it will
    normally use the default value of {\attval{normal}{variant}{p-{\codeml}}},
    even if it was contained in a larger element that set this attribute to a
    different value. For attributes like this, specifying a value with an
    {\element{cstyle}} element has the effect of changing the default value for
    all elements within its scope.  The net effect is that setting the attribute
    value with {\element{cstyle}} propagates the change to all the elements it
    contains directly or indirectly, except for the individual elements on which
    the value is overridden. Unlike in the case of inherited attributes, elements
    that explicitly override this attribute have no effect on this attribute's
    value in their children.
\end{itemize}

Note that attribute values inherited from an {\element{cstyle}} in any manner
affect a given element in the {\element{cstyle}}'s content only if that attribute
is not given a value in that element's start tag. On any element for which the
attribute is set explicitly, the value specified on the start tag overrides the
inherited value.

As stated above, {\element{style}} accepts all attributes of all {\codeml}
presentation elements which do not have required values. That is, all attributes
which have an explicit default value or a default value which is inherited or
computed are accepted by the mstyle element. 

\section{Content Elements}\label{sec:content}

The content fragment of {\codeml} is used to represent the abstract structure of
the algorithm underlying the program and unambiguously identify the symbols used
in it, independently of the surface form. We will again use the {\java} code
fragment for the {\ttin{find}} procedure in {\mylstref{find-raw}} as a running
example in this section. The content-{\codeml} equivalent of this is given in 
{\mylstref{find-content}}.

\begin{lstlisting}[float,frame=lines,label=lst:find-content,escapechar=\%,
   caption={The {\codeml} content for the code snippet in {\mylstref{find-raw}}},
   index={[2]ccdef,symbol,type,apply,bind,bvar,ccb,ccv}]
 <ccdef export="find">
   <ccsym cd="java.dec" name="public-type-function"/>
   <apply>
     <ccsym cd="java.types" name="funtype"/> %\label{lst-cr:lst:find-content:ccsym}%
     <ccsym cd="java.types" name="int"/>
     <ccsym cd="java.types" name="int"/>
   </apply>
   <bind>
     <ccsym cd="java.proc" name="function"/> %\label{lst-cr:lst:find-content:ccsym-bind}%
     <bvar><ccv name="x"/></bvar>            %\label{lst-cr:lst:find-content:bvar}%
     <apply>                                 %\label{lst-cr:lst:find-content:body}%
       <ccsym cd="java.control" name="if"/>
       <apply>
         <ccsym cd="java.arith" name="less"/>
         <apply>
           <ccsym cd="java.array" name="select"/>
           <ccv name="s"/>                   %\label{lst-cr:lst:find-content:ccv}%
           <ccv name="x"/>
         </apply>
         <ccb type="number">0</ccb>          %\label{lst-cr:lst:find-content:ccb}%
       </apply>
       <apply>
         <ccsym cd="java.proc" name="return"/>
         <ccv name="x"/>
       </apply>
       <apply>
         <ccsym cd="java.proc" name="return"/>
         <apply>
           <ccsym cd="union-find" name="find"/>
           <apply>
             <ccsym cd="java.array" name="select"/>
             <ccv name="s"/>
             <ccv name="x"/>
           </apply>
         </apply>
       </apply>
     </apply>
   </bind>
 </ccdef>
\end{lstlisting}

The {\codeml} content elements only have the common {\codeml}
attributes\ednote{crossref, define them somewhere above} attributes, and an
optional {\attribute{id}{c-{\codeml}}} attribute, which can be used for
cross-referencing by by the {\ttin{xlink}} attributes by the
presentation-{\codeml} elements (see the introduction to
section~\ref{sec:presentation} and section~\ref{sec:integration}).

\begin{myfig}{content}{The {\codeml} Content Elements}
  \quicktable{\contenttable{}}
\end{myfig}

\subsection{Token Elements}\label{sec:content:token}

The {\codeml} {\indextoo{token element}s}\index{element token} are similar in
nature to the {\openmath} elements, we have:
\begin{description}
\item[{\eldef{ccv}}] for {\indextoo{variable}s}: this element has a required
  attribute {\attribute{name}{ccv}} for the name of the variable. Since for
  variables the name determines the content, this element is empty.
\item[{\eldef{ccsym}}] for {\indextoo{defined symbol}s}\index{symbol!defined}:
  this element is empty as well and has two required attributes
  {\attribute{name}{ccsym}} for the name of the symbols, and
  {\attribute{cd}{ccsym}} for the {\index{content
      dictionary}}\index{dictionary!content} that describes, implements, or
  defines the meaning of the symbol.  These two attributes totally determine the
  content of the symbol (by reference to a defining document\ednote{say some more
    about the module system, ...  later}), so the {\element{ccsym}} element is
  empty.
\item[{\eldef{ccb}}] that contains {\indextoo{basic data}}\index{data!basic},
  such as {\indextoo{number}s}, {\indextoo{string}s}, etc. The type of this data
  can be specified in the optional attribute {\attribute{type}{ccb}}\ednote{do we
    want to enumerate the basic data types, or leave them open for the user?}
\end{description}

\subsection{Complex Elements}\label{sec:content:complex}

Content-{\codeml} knows the follwing three {\indextoo{complex
    element}s}\index{element complex}, which allow to combine content-{\codeml}
expressions to larger ones.

\begin{description}
\item[{\eldef{apply}}] for (function/procedure)
  application\index{function!application}\index{application!function}.  This
  element is a constructor for function- and {\indextoo{procedure
      call}s}\index{function call}\index{call!function/procedure}: The first child
  is interpreted as a function or procedure and is given the remaining children of
  the {\element{apply}} element as argument.
\item[{\eldef{bind}}] this {\indextoo{binding
      constructor}}\index{contructor!binding} {\element{bind}} is used to
  represent {\indextoo{binding}} and {\indextoo{abstraction}} constructions. It is
  primarily used in function and {\indextoo{procedure
      definition}s}\index{function!definition}\index{definition!function/procedure}.
  The first child of this a {\codeml} binding expression must be a
  {\element{ccsym}} element, which defines a binding symbol. The second child must
  be a {\eldef{bvar}} element, which contains {\element{ccv}} elements for all
  the {\indextoo{formal parameter}s}\index{parameter!formal} of the binding. The
  third child of the {\element{bind}} element is known as the {\indextoo{body}}.
  It is an arbitrary content-{\codeml} element, which may (but in general need
  not) contain occurrences of the bound variables. Note that the bound variables
  specified in the {\element{bvar}} element only have or {\index{scope}} in the
  body. An occurrence of an {\element{ccv}} element outside the body is
  semantically a different variable, even if it carries the same name.
\item[{\eldef{ccdef}}] for definitions and declarations: In contrast to the
  {\element{bind}} element, this one defines a set of symbols to be visible
  outside its body: Its required {\attribute{exports}{ccdef}} attribute specifies
  a set of symbol names as a whitespace-separated list of names. 
  
  The first child of the {\element{ccdef}} element is a {\indextoo{declaration
      symbol}}\index{symbol!declaration} (specified as a {\element{ccsym}} element
  that specifies the {\indextoo{definition schema}}\index{schema!definition}.  For
  instance, in the {\element{ccdef}} element in {\mylstref{find-content}},
  this is the symbol
  \begin{quote}
    {\lstinline[basicstyle=\small]{<ccsym cd="java.dec" name="public-type-function"/>}}     
  \end{quote}
  that specifies that the second child is a {\indextoo{type}} (the type of the
  {\indextoo{defined symbol}}\index{symbol!defined}), and the third child is a
  representation of a function (in this case via a {\element{bind}} element).
\end{description}

\section{Interaction of Content and Presentation}\label{sec:integration}

\begin{newpart}{copied from MathML, rework heavily}
In the last sections we have seen two styles of markup for program code: 

{\em {\indextoo{Presentation markup}}\index{markup!presentation}} captures
{\indextoo{notational structure}}\index{structure!notational}. It encodes the
notational structure of an expression in a sufficiently abstract way to facilitate
rendering to various media. It does this by providing information such as
structured grouping of expression parts, classification of symbols, etc.

Presentation markup does not directly concern itself with the mathematical
structure or meaning of the code. In many situations, notational structure and
mathematical structure are closely related, so a sophisticated processing
application may be able to heuristically infer mathematical meaning from
notational structure, provided sufficient context is known. However, in practice,
the inference of mathematical meaning from mathematical notation must often be
left to the reader. As a consequence, employing presentation tags alone may limit
the ability to re-use a {\codeml} object in another context, especially evaluation
by external applications.

{\em {\indextoo{Content markup}}\index{markup!content}} captures {\indextoo{mathematical
    structure}}\index{structure!mathematical}. It encodes mathematical structure
in a sufficiently regular way in order to facilitate the assignment of
mathematical meaning to an expression by application programs.  Though the details
of mapping from mathematical expression structure to mathematical meaning can be
extremely complex, in practice, there is wide agreement about the conventional
meaning of many basic mathematical constructions.  Consequently, much of the
meaning of a content expression is easily accessible to a processing application,
independently of where or how it is displayed to the reader. In many cases,
content markup could be cut from a Web browser and pasted into a mathematical
software tool with confidence that sensible values will be computed.

Since content markup is not directly concerned with how an expression is
displayed, a renderer must infer how an expression should be presented to a
reader. While a sufficiently sophisticated renderer and style sheet mechanism
could in principle allow a user to read mathematical documents using personalized
notational preferences, in practice, rendering content expressions with notational
nuances may still require intervention of some sort. As a consequence, employing
content tags alone may limit the ability of the author to precisely control how an
expression is rendered.

Both {\indextoo{content}} and {\indextoo{presentation}} tags are necessary in
order to provide the full expressive capability one would expect in a markup
language for program code.  {\codeml} offers authors elements for both content and
presentation markup.  Whether to use one or the other, or a combination of both,
depends on what aspects of rendering and interpretation an author wishes to
control, and what kinds of re-use he or she wishes to facilitate.
\end{newpart}

In many common situations, an author or authoring tool may choose to generate
either presentation or content markup exclusively. Some applications however are
able to make use of both presentation and content information. For these
applications it is desirable to provide both forms of markup for the same
mathematical expression.

\begin{myfig}{semantics}{The {\codeml} Interaction Elements}
  \quicktable{\semtable{}}
\end{myfig}

To combine {\indextoo{presentation markup}}\index{markup!presentation} and
{\indextoo{content markup}}\index{markup!content}, {\codeml} provides the
{\element{semantics}} element. The first child of this element is a content
{\codeml} expression, and remaining children are {\eldef{pcode}} or
{\eldef{rawcode}} elements. The {\element{pcode}} elements contain
presentation-{\codeml} representations of the program in the first child, and the
{\element{rawcode}} elements raw code, i.e.  code fragments that have not been
marked up in {\codeml}. The {\attribute{format}{pcode,rawcode}} specifies the
programming language, the code fragment is in. Note that it is an error to have
more than one {\element{pcode}} or {\element{rawcode}} element with the same
{\attribute{format}{pcode,rawcode}} attribute. As the example in
{\mylstref{find-combined}} shows, it is allowed to have {\element{pcode}} and
{\element{rawcode}} elements of the same {\attribute{format}{pcode,rawcode}}.

\begin{lstlisting}[float,frame=lines,label=lst:find-combined,
   language=java,numbers=none,escapechar=\%,
   caption={Conbining markup Styles},
   index={find}]
<semantics>
  <ccdef export="find"><!--%\ldots see {\mylstref{find-content}} \ldots%--></ccdef>
  <pcode format="pseudo"><cpg> <!--%\ldots see {\mylstref{find-pseudo}} \ldots%--></cpg></pcode>
  <pcode format="java"><cpg> <!--%\ldots see {\mylstref{find-presentation}} \ldots%--></cpg></pcode>
  <rawcode format="java"><!--%\ldots see {\mylstref{find-raw}} \ldots%--></rawcode>
</semantics>   
\end{lstlisting}

In contrast to representation formats like {\mathml}, which allow mixing
presentation and content at arbitrary levels, the {\codeml} format only allows
this at the top-level.  This is called {\indextoo{parallel
    markup}}\index{markup!parallel}.

\subsection{Parallel Markup by Cross-references}

Top-level pairing of independent presentation and content markup is sufficient for
many, but not all, situations. Applications that allow treatment of
sub-expressions of mathematical objects require the ability to associate
presentation, content or information with the parts of an object with mathematical
markup. Top-level pairing with a {\eldef{semantics}} element is insufficient in
this type of situation; identification of a sub-expression in one branch of
{\element{semantics}} element gives no indication of the corresponding parts in
other branches.
  
To better accommodate applications that must deal with sub-expressions of large
objects, {\codeml} uses cross-references between the branches of a
{\element{semantics}} element to identify corresponding sub-structures.

Cross-referencing is achieved using {\attribute{id}{c-{\codeml}}} and
{\attribute{xlink:xref}{p-{\codeml}}} attributes within the branches of a
containing semantics element. These attributes may optionally be placed on MathML
elements of any type.

An {\attribute{id}{c-{\codeml}}} attribute and a corresponding
{\attribute{xlink:xref}{p-{\codeml}}} appearing within the same
{\element{semantics}} element create a correspondence between sub-expressions.
  
In general, there will not be a one-to-one correspondence between nodes in
parallel branches. For example, a presentation tree may contain elements, such as
parentheses, that have no correspondents in the content tree. Therefore {\codeml}
puts the {\attribute{id}{c-{\codeml}}} attributes on the branch with the content
tree, which has the finest-grained node structure and the
{\attribute{xlink:xref}{p-{\codeml}}} attributes on the presentation-branches.  

Let us fortify our intuition about this with an extended development of the
{\ttin{find}} procedure that combines all the material discussed above.

\def\boxit#1{\fbox{#1}}
{\scriptsize\lstinputlisting[
  index={find,cpg,cpo,cpg,cptype,code,cpc,cpd,cpt,ccdef,symbol,type,apply,bind,bvar,ccb,ccv},
  emph={id,xlink:xref},escapebegin={<!--+},escapeend={+-->}]
{../../examples/spec/find.cml}}

\ref{lst-cr:fig:find-ccode:ccsym-bind}
%%% Local Variables: 
%%% mode: latex
%%% TeX-master: "codeml"
%%% End: 

\svnInfo $Id: concl.tex 15 2008-02-22 07:55:51Z kohlhase $
\svnKeyword $HeadURL: https://svn.omdoc.org/repos/codeml/doc/spec/concl.tex $
\chapter{Conclusion}\label{sec:concl}\label{sec:future}
\ednote{say something about what we have done}
\ednote{say something about why this is useful}

%%% Local Variables: 
%%% mode: latex
%%% TeX-master: "codeml"
%%% End: 

\newpage 
\bibliographystyle{alpha} 
\bibliography{codeml} 
\newpage
{\small\input{\jobname.ind}}
\svnInfo $Id: appendix.tex 15 2008-02-22 07:55:51Z kohlhase $
\svnKeyword $HeadURL: https://svn.omdoc.org/repos/codeml/doc/spec/appendix.tex $
\begin{appendix}

\def\contentcat{Content}
\def\cptoken{PresToken}
\def\cpgroup{PresGroup}
\def\cptext{PresText}
\def\intercat{Inter}
\chapter{Quick-Reference Table to the {\codeml} Elements}\label{sec:table}
\def\tabelt#1#2#3#4#5{{#1}&\pageref{eldef:#1}&{#2}&{#3}&{#4}&{#5}\\\hline}
{\scriptsize
\begin{longtable}{|>{\tt}p{2.1cm}|l|l|>{\tt}p{2cm}|>{\tt}p{2cm}|>{\tt}p{3cm}|}\hline
{\rm Element}& p. & Type  & {\rm Required}  & {\rm Optional} &  Content \\\hline
             & &        & {\rm Attribs}  & {\rm Attribs} &         \\\hline\hline
\tabelt{apply}\contentcat{}{id}{(apply| bind| bvar| ccv| ccb| ccsym)*}
\tabelt{bind}\contentcat{}{id}{ccsym, bvar, (apply| bind| bvar| ccv| ccb| ccsym)}
\tabelt{bvar}\contentcat{}{id}{ccv*}
\tabelt{ccdef}\contentcat{export}{id}{ccsym, (apply| bind| bvar| ccv| ccb| ccsym)}
\tabelt{ccsym}\contentcat{cd, name}{id}{EMPTY}
\tabelt{ccv}\contentcat{name}{id}{EMPTY}
\tabelt{cpb}\cptoken{}{id, xlink:*, type, variant, style, color, background}{CDATA}
\tabelt{cpbr}\cpgroup{}{id, xlink:*}{CDATA}
\tabelt{cpc}\cptext{}{}{cpt*}
\tabelt{cpd}\cptext{}{}{cpt*}
\tabelt{cpg}\cpgroup{}{id, xlink:*, open, close, separator, indent, breaks}  
      {(cpg| cpb| cpo| cpi| cpbr| cptype| cpd| cpc |cpr | cpstyle)*}
\tabelt{cpi}\cptoken{}{id, xlink:*, variant, style, color, background}{CDATA}
\tabelt{cpo}\cptoken{}{id, xlink:*, type, variant, style, color, background}{CDATA}
\tabelt{cpt}\cptext{}{variant, style, color, background}{CDATA}
\tabelt{cptype}\cptoken{}{id, xlink:*, variant, style, color, background}{CDATA}
\tabelt{pcode}\intercat{format}
      {type, href,id, xlink:*, open,close, separator, indent, breaks}
      {(cpg| cpb| cpo| cpi| cpbr| cptype| cpd| cpc |cpr | cpstyle)*}
\tabelt{rawcode}\intercat{format}{type, href}{PCDATA}
\tabelt{semantics}\intercat{}{}
      {(apply| bind| bvar| ccv| ccb| ccsym, ccdef)*,(pcode| rawcode)*}
\end{longtable}}

\chapter{Quick-Reference Table to the {\codeml} Attributes}\label{sec:att-table}
\def\atabelt#1#2#3#4{{#1}&{#2}&{#3}\\\hline&\multicolumn{2}{|p{9cm}|}{#4}\\\hline\hline}
{\footnotesize\begin{longtable}{|>{\tt}p{2.5cm}|>{\tt}p{4cm}|>{\tt}p{5cm}|}\hline
{\rm Attribute} & {\em element} & Values \\\hline\hline
\atabelt{breaks}{cpg}{bob-bcb, ob-bcb, bo-bcb, o-bcb, bob-cb, ob-cb, bo-cb, o-cb,
  bob-bc, ob-bc, bo-bc, o-bc, bob-c, ob-c, bo-c, o-c}
  {SAY SOMETHING HERE, also about prioritized breaks}
\atabelt{cd}{ccsym}{{\rm theory/module name}}{The name of a theory or a module
  that exports this symbol}
\atabelt{close}{cpg}{string}{The closing bracket of group of {\codeml} constructs
  this is inserted into the presentation in place of the opneing tag of its
  {\element{cpg}} element}
\atabelt{color, background}{p-{\codeml}}{{\#{\sl rgb}},{\#{\sl rrggbb}} aqua,
  black, blue, fuchsia, gray, green, lime, maroon, navy, olive, purple, red,
  silver, teal, white, yellow}
  {The color of the text representation or the background of a
   presentation-{\codeml} element}
\atabelt{exports}{ccdef}{string}{the name of a symbol defined by the {\element{ccdef}}
  element}
\atabelt{format}{pcode,rawcode}{}{The format, e.g. the programming language}
\atabelt{id}{*}{}{a string that identifies the element}
\atabelt{indent}{cpg}{number}{a factor for the default indentation
  increment used by the presentation agent}
\atabelt{name}{ccsym}{string}{The name the symbol}
\atabelt{open}{cpg}{string}{The opening bracket of group of {\codeml} constructs;
  this is inserted into the presentation in place of the opneing tag of its
  {\element{cpg}} element}
\atabelt{separator}{cpg}{string}{The separator bracket of group of {\codeml} constructs
  this is inserted into the presentation between the children of its
  {\element{cpg}} element}
\atabelt{size}{p-{\codeml}}{small, normal, big, {\rm number}, {\rm v-unit}}
  {The font size of the text representation of a presentation-{\codeml} element}
\atabelt{type}{cpb, ccb}{number}{the type of the basic object}
\atabelt{type}{cpo}{built-in, imported, defined}{the type of the basic object}
\atabelt{variant}{p-{\codeml}}{normal, bold, italic, bold-italic, double-struck,
  bold-fraktur, script, bold-script, fraktur, sans-serif, bold-sans-serif,
  sans-serif-italic, sans-serif-bold-italic, monospace}
  {The font variant of the text of a presentation-{\codeml} element}
\atabelt{xlink:*}{cp*}{} {a crossreference to a semantically equivalent content
  element; the value of {\attribute{xlink:xref}{presentation-CodeML}} (an URIRef)
  specifies the element, the value of
  {\attribute{xlink:type}{presentation-CodeML}} must be {\ttin{simple}} (it is
  fixed in the DTD), and the {\attribute{xlink:role}{presentation-CodeML}} is a
  fixed URL to a document that explains this link type.}
\atabelt{xml:lang}{cpt}{ISO 639: en, de, it, ko, ...}{the language the text is
  written in}
\end{longtable}}

\chapter{The {\codeml} RelaxNG Schema}\label{sec:rnc}
We reprint the current version of the {\codeml} RelaxNG schema. The original can be found
at {\url{https://svn.omdoc.org/repos/codeml/RelaxNG}}

\section{The {\codeml} Schema Driver}
\lstinputlisting[language=RNC,nolol]{../../RelaxNG/codeml.rnc}

\section{The {\codeml} Module PRES}
Module PRES introduces the presentation {\codeml} elements. 

\lstinputlisting[language=RNC,nolol,
                 index={cpg,cpb,cpo,cpi,cpbr,cptype,cpt,cpd,cpc,cpstyle}]
                {../../RelaxNG/codeml-pres.rnc}

\section{The {\codeml} Module CONT}
Module CONT introduces the content {\codeml} elements. 

\lstinputlisting[language=RNC,nolol,
                 index={apply,bind,bvar,ccv,ccb,ccsym,ccdef,symbol,type}]
                {../../RelaxNG/codeml-cont.rnc}

\section{The {\codeml} Module INTER}
Module INTER introduces elements for the interaction of content and presentation
{\codeml} elements. The top-level {\element{code}} element is among then. 

\lstinputlisting[language=RNC,nolol,index={semantics,pcode,rawcode,code}]
{../../RelaxNG/codeml-inter.rnc}

This schema includes the generic schema for Dublin Core Metadata and for the MARC relator 
set, we include both of them for reference
\subsection{The Dublin Core Schema}
\lstinputlisting[language=RNC,nolol]{../../RelaxNG/dublincore.rnc}
\subsection{The MARC Relators Schema}
\lstinputlisting[language=RNC,nolol]{../../RelaxNG/MARCRelators.rnc}
\end{appendix}
              
%%% Local Variables: 
%%% mode: latex
%%% TeX-master: "codeml"
%%% End: 

\ednotemessage
\end{document}

%%% Local Variables: 
%%% mode: latex
%%% TeX-master: t
%%% End: 
