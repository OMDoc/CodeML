\svnInfo $Id: intro.tex 15 2008-02-22 07:55:51Z kohlhase $
\svnKeyword $HeadURL: https://svn.omdoc.org/repos/codeml/doc/spec/intro.tex $
\chapter{Introduction}\label{sec:intro} 

This specification defines the {\indextoo{Code Markup
    Language}}\index{markup!language! for program code}, or {\codeml}.  {\codeml} is an
{\xml} application for describing program code and capturing both its structure and
content. The goal of {\codeml} is to enable program code to be served, received,
and processed on the World Wide Web, just as HTML has enabled this functionality
for text.
  
The {\em primary purpose} of {\indextoo{program}} code\index{code!program} is as a
means of communication between the {\indextoo{programmer}} and the machine as a
vehicle to express programs that can be executed by the machine. For this purpose,
the field of {\indextoo{computer science}} has developed a multitude of
{\indextoo{programming language}s}, i.e. {\indextoo{formal
    language}s}\index{language!formal} with a well-defined structure and fixed
semantics that allow the programmer to express her operational intentions as
{\indextoo{texts}} called {\indextoo{program code}}. This code is then interpreted
by a suitable programs (e.g. a {\indextoo{compiler}}) and translated into native
instructions to the computer, which are then executed to achieve the desired
result.

An {\em auxiliary purpose} of program code is to communicate about
{\indextoo{algorithm}s} and programs between humans, e.g. in {\indextoo{program
    development}}\index{development!of programs} {\indextoo{documentation}},
{\indextoo{quality assurance}}, or computer science {\indextoo{education}}. To
allow humans to cope with the complexities of program code, the program
development process is supported by a variety of ``intelligent development
environments''\index{development!environment}, which provide features like code
{\indextoo{pretty-printing}} (e.g. syntax {\indextoo{highlighting}},
{\indextoo{indentation}}), {\indextoo{hyper-linking}}, and even higher-level
techniques like {\indextoo{literate programming}}, or {\indextoo{documentation}}
{\indextoo{extraction}} from {\indextoo{commented code}}\index{code!commented}.

Unfortunately, this functionality is directly tied to the development environment
and (usually) to a particular programming language: The program text has to be
analyzed, and the structural and syntactic information has to be conveyed to the
user in a dedicated user interface. It is currenly only possible to export the
functionality of these development environments to the World Wide Web to a very
limited extent. 
\begin{itemize}
\item Web markup formats like {\html} only provide limited primitives for
  pretty-printing program listings; the possibilities are largely limited to using
  the
  {\tt{pre}}\index{element!{\tt{pre}}(html)}\index{html!{\tt{pre}}}\index{pre@{\tt{pre}}
    (html)} element, {\indextoo{table}s}, or non-breaking spaces {\ttin{\&nbsp;}}.
  Together with CSS styling this is used for (static, fixed-width) indenting and
  syntax high-lighting.
\item Using technologies like {\xsl} formatting objects (which allow
  flexible-width formatting\ednote{check that they do}) is problematic, since the
  syntactic and structural information necessary for pretty-printing is implicit
  in the program structure and a full parser that is needed to reveal it,
  transcends the possibilities of both {\ttin{javascript}} and and {\xslt} (which
  are the computational engines built into current browsers). Moreover, an
  approach based on parsing programming languages would necessitate the
  availability of gramars for all programming languages. 
\item As programming languages are optimized for compilation and execution, they
  do not supply an infrastructure for referencing program fragments, which would
  be\ednote{continue}
\end{itemize}

\ednote{say something more, dynamics, local services without
  program interpretation. This is similar to the situation to math formulae, say
  something about what we can do with {\mathml}}

The concrete design of the {\codeml} format is inspired by the
{\mathml}\index{mathml@{\mathml}} (the {\indextoo{Mathematical Markup Language}})
and {\openmath} format for markup of mathematical formulae\index{markup!language!
  for mathematical formulae}\index{formulae!mathematical}\index{mathematical
  formulae} and can be used in conjuction with these formats in a variety of
document formats.


The most relevant related work is probably the {\tt{listings.sty}}
package~\cite{Heinz:tlp02} for {\LaTeX}, which allows to mark up program listings
for print presentation.\ednote{say some more, maybe move somewhere else}

\ednote{given an overview over what there already is.}

%%% Local Variables: 
%%% mode: latex
%%% TeX-master: "codeml"
%%% End: 
